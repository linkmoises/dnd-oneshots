\documentclass[10pt,twoside,twocolumn,openany]{dndbook}

\usepackage[spanish]{babel}
\usepackage[utf8]{inputenc}

\title{Luna de sangre\\
\large One shot para personajes de nivel 1}
\author{linkmoises}
\date{\today}

\begin{document}

\frontmatter

\maketitle

%% \mainmatter%

\part{Campamento Goblin}

\chapter{Luna de Sangre}

\DndDropCapLine{L}{as niñas de una aldea cerca} de Phandalin han estado desapareciendo en los últimos 7 días. Nadie sabe que está ocurriendo, solo se sabe que se acuestan y a la mañana siguiente ya no están. Los padres y madres preocupados ahora duermen en la misma habitación de sus hijos o les amarran a la cama al momento de dormir para evitar que desaparezcan. La última niña en desaparecer fue la hija del gobernador hace solo un día. Han desaparecido un total de 12 niñas en ese lapso de tiempo.

Desde entonces, el gobernador, Tormund Esses, ha colgado carteles por toda la aldea y en Phandalin ofreciendo recompensas por quién pueda dar información sobre el paradero de los niños o los pueda encontrar.

\section{Presentación}

Los aventureros son aprendices de héroes que intentan ganar algo de fama y dinero. Varios aventureros se presentaron el último día para enfrentar el reto y encontrar a los niños. Por desgracia, ya sea por mala suerte o inexperiencia, fueron derrotados estrepitosamente cuando se enfrentaron a ellos y han sido hechos prisioneros, quién sabe con qué oscuro propósito.

Los aventureros despertarán en celdas a la intemperie en la mitad de la noche, hechas por vigas de madera de no menos de 8 pulgadas de diámetro, 9 pies de alto y con un espacio de 6' x 6'. Las vigas han sido encebadas para evitar que algún habilidoso logré trepar por ellas y terminan en puntas afiladas.

Queda a discreción del Master decidir cómo fueron capturados los jugadores y en qué momento. A lo mejor estaban explorando el bosque de manera individual y fueron emboscados o cayeron en alguna trampa y fueron capturados. Lo común en todos estos casos es que llegaron inconscientes a las celdas dónde los tienen prisioneros.

\section{Parte 1: Campamento Goblin}

A pesar de preferir ambientes oscuros y húmedos como cuevas, por alguna razón diferente a la usual un grupo de 20 goblins han establecido un campamento en las profundidades del bosque rodeado de árboles altos. Hay 5 tiendas de campaña con una central más del doble de tamaño que las otras. La explanada ofrece un terreno bastante llano y han colocado ocho torres vigías alrededor del campamento. Las tres torres vigías más cercanas a las celdas improvisadas tienen un arquero cada una. Las otras torres vigías las alternan turnos goblins arqueros de manera que por momentos puede haber 2 o 3 de ellas ocupadas.

Las celdas están ubicadas al sur del campamento y en su mayoría tienen 2 niñas por cada celda. Niñas asustadas en su mayoría, agotadas por la mala alimentación y deshidratadas algunas por qué están presentando cuadros de vómitos y diarrea.

\begin{DndComment}{Prisioneros}
Los jugadores estarán repartidos de manera individual en cada celda. Cuando empiecen a preguntar qué ha pasado o donde están, podrán conseguir la siguiente información de un prisionero (perteneciente a la aldea) que llegó el día anterior al grupo.

\begin{itemize}
  \item Todas las niñas desaparecidas están repartidas en las celdas y todas tienen 12 años.
  \item Los miembros de la banda se escabullen en la noche y secuestran a las niñas en sus casas en la aldea cercanas.
  \item Un hombre encapuchado con la piel de color naranja u ocre dirige a los goblins. Su nombre es Gheed y dirige a los goblins con mano de hierro manteniendolos aterrorizados.
  \item Gheed tiene una mascota, Geb, un perro del inframundo de dos cabezas, que alimenta con carne humana (y también de goblins). El chamán hobgoblin lo usa para intimidar a los goblins y a las prisioneras.
  \item Gheed se la pasa hablando de una luna de sangre que ocurrirá a medianoche la noche siguiente.
\end{itemize}

\end{DndComment}

\DndArea{Tienda principal}
Se encuentra en el centro del campamento, tiene una sola entrada y salida. Es custodiada siempre por 2 goblins. En su interior, el piso es de piel de oso pardo y con una silla de madera tallada a manera de trono. Hay una cama enorme y lujosa para las condiciones de la tienda, varios alijos de madera y un arcón de hierro dónde guarda sus tesoros robados más importantes. Llama la atención una jaula de acero inmediatamente a la derecha luego de la entrada, dónde se encuentra Geb.

\DndArea{Tiendas goblins}
Son 3 en total, bastante sencillas son dormitorios para los goblins, hay varios catres, una mesa donde colocan su equipo y varios barriles.

\DndArea{Tienda del custodio}
El goblin que custodia las celdas es también el goblin que se encarga de cocinar, pues el que menos trabajo hace si solamente se dedicará a custodiar las celdas. Vigila que los prisioneros no hagan ruido. Una o dos veces al día hace un recorrido por las celdas. Diariamente los goblins rotan este puesto de manera que cada día hay uno diferente en este puesto. Hay varias mesas y provisiones para cocinar.

\DndArea{Torres vigías}
Estructuras hechas con vigas de madera que se levantan hasta cinco metros del suelo. Permiten a los goblins arqueros vigilar la explanada del campamento.

\end{document}
